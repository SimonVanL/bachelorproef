%==============================================================================
% Sjabloon onderzoeksvoorstel bachelorproef
%==============================================================================
% Gebaseerd op LaTeX-sjabloon ‘Stylish Article’ (zie voorstel.cls)
% Auteur: Jens Buysse, Bert Van Vreckem

\documentclass[fleqn,10pt]{voorstel}

%------------------------------------------------------------------------------
% Metadata over het voorstel
%------------------------------------------------------------------------------

\JournalInfo{HoGent Bedrijf en Organisatie}
\Archive{Bachelorproef 2018 - 2019} % Of: Onderzoekstechnieken

%---------- Titel & auteur ----------------------------------------------------

\PaperTitle{Gezichtsherkenning als basis voor een betere lesomgeving}
\PaperType{Onderzoeksvoorstel Bachelorproef} % Type document

\Authors{Simon Van Lombergen\textsuperscript{1}} % Authors
\CoPromotor{Piet Pieters\textsuperscript{2} (Bedrijfsnaam)}
\affiliation{\textbf{Contact:}
  \textsuperscript{1} \href{mailto:simon.vanlombergen.y9313@student.hogent.be}{simon.vanlombergen.y9313@student.hogent.be};
  \textsuperscript{2} \href{mailto:piet.pieters@acme.be}{piet.pieters@acme.be};
}

%---------- Abstract ----------------------------------------------------------

\Abstract{Hier schrijf je de samenvatting van je voorstel, als een doorlopende tekst van één paragraaf. Wat hier zeker in moet vermeld worden: \textbf{Context} (Waarom is dit werk belangrijk?); \textbf{Nood} (Waarom moet dit onderzocht worden?); \textbf{Taak} (Wat ga je (ongeveer) doen?); \textbf{Object} (Wat staat in dit document geschreven?); \textbf{Resultaat} (Wat verwacht je van je onderzoek?); \textbf{Conclusie} (Wat verwacht je van van de conclusies?); \textbf{Perspectief} (Wat zegt de toekomst voor dit werk?).

Bij de sleutelwoorden geef je het onderzoeksdomein, samen met andere sleutelwoorden die je werk beschrijven.

Vergeet ook niet je co-promotor op te geven.
}

%---------- Onderzoeksdomein en sleutelwoorden --------------------------------
%Sleutelwoorden:
%
% Het eerste sleutelwoord beschrijft het onderzoeksdomein. Je kan kiezen uit
% deze lijst:
%
% - Mobiele applicatieontwikkeling
% - Webapplicatieontwikkeling
% - Applicatieontwikkeling (andere)
% - Systeembeheer
% - Netwerkbeheer
% - Mainframe
% - E-business
% - Databanken en big data
% - Machineleertechnieken en kunstmatige intelligentie
% - Andere (specifieer)
%
% De andere sleutelwoorden zijn vrij te kiezen

\Keywords{Machineleertechnieken en kunstmatige intelligentie. Gezichtsherkenning --- Onderwijs} % Keywords
\newcommand{\keywordname}{Sleutelwoorden} % Defines the keywords heading name

%---------- Titel, inhoud -----------------------------------------------------

\begin{document}

\flushbottom % Makes all text pages the same height
\maketitle % Print the title and abstract box
\tableofcontents % Print the contents section
\thispagestyle{empty} % Removes page numbering from the first page
%------------------------------------------------------------------------------
% Hoofdtekst
%------------------------------------------------------------------------------

% De hoofdtekst van het voorstel zit in een apart bestand, zodat het makkelijk
% kan opgenomen worden in de bijlagen van de bachelorproef zelf.
%---------- Inleiding ---------------------------------------------------------

\section{Introductie} % The \section*{} command stops section numbering
\label{sec:introductie}

% Hier introduceer je werk. Je hoeft hier nog niet te technisch te gaan.
% Je beschrijft zeker:
% \begin{itemize}
%  \item de probleemstelling en context
%  \item de motivatie en relevantie voor het onderzoek
%  \item de doelstelling en onderzoeksvraag/-vragen
% \end{itemize}

Het aanbrengen van nieuwe leerstof is een continue opdracht voor lectoren. Tijdens een les kan een lector een indicatie krijgen van zijn studenten of ze geengageerd zijn en gefocust zijn maar wat als een datagedreven model hierbij zou kunnen helpen? Technologische vooruitgang op het vlak van machineleertechnieken en artificiele intelligentie zorgen ervoor dat lectoren meer inzicht kunnen hebben over de graad van focus van de studenten.

Dit onderzoek tracht antwoorden te vinden op de volgende onderzoeksvragen: 
\textbf{Hoe kan gezichtsherkenning van studenten via machineleertechnieken de leskwaliteit verbeteren in het hoger onderwijs?}
\begin{itemize}
    \item Welke software is er geschikt om focus te meten op vlak van emoties?
    \item Hoe kunnen lectoren hun les verbeteren wetende dat de focus laag is?
    \item Kan een gemeten focustrend de leskwaliteit daadwerkelijk verbeteren?
\end{itemize}

%---------- Stand van zaken ---------------------------------------------------

\section{State-of-the-art}
\label{sec:state-of-the-art}

%Hier beschrijf je de \emph{state-of-the-art} rondom je gekozen onderzoeksdomein. Dit kan bijvoorbeeld een literatuurstudie zijn. Je mag de titel van deze sectie ook aanpassen (literatuurstudie, stand van zaken, enz.). Zijn er al gelijkaardige onderzoeken gevoerd? Wat concluderen ze? Wat is het verschil met jouw onderzoek? Wat is de relevantie met jouw onderzoek?
%
%Verwijs bij elke introductie van een term of bewering over het domein naar de vakliteratuur, bijvoorbeeld! Denk zeker goed na welke werken je refereert en waarom.

% Voor literatuurverwijzingen zijn er twee belangrijke commando's:
% \autocite{KEY} => (Auteur, jaartal) Gebruik dit als de naam van de auteur
%   geen onderdeel is van de zin.
% \textcite{KEY} => Auteur (jaartal)  Gebruik dit als de auteursnaam wel een
%   functie heeft in de zin (bv. ``Uit onderzoek door Doll & Hill (1954) bleek
%   ...'')

%Je mag gerust gebruik maken van subsecties in dit onderdeel.

\subsection{Machineleertechnieken en gezichtsherkenning}
De thesis \textcite{Jonsson2018} geeft duiding over het onderzoeksdomein machineleertechnieken en kunstmatige intelligentie. Kunstmatige intelligentie (AI) is concept bestaande van 1950 dat in de praktijk als volgt beschreven kan worden: het automatiseren van taken die menselijk denkwerk nodig hebben. 

Een huidige vorm hiervan zijn machineleertechnieken (ML) waaronder gezichts- en emotieherkenning vallen. De aanpak bij ML bestaat niet uit klassieke geprogrammeerde regels. Een systeem op basis van ML tracht een regelmaat te vinden in een gestructureerde dataset (voor dit onderzoek gefilmde gezichten) en past bekomen regels toe op nieuwe data. 

\subsection{Gerelateerde onderzoeken}
Er zijn veel onderzoeken gedaan naar emotiedetectie via ML. Deze beschrijven onder andere verdriet, woede, afkeer, schrik, blijheid, verrassing. Deze geven inzicht voor het opstellen van een soortgelijke proefopstelling en mogelijke conclusies achteraf. \textcite{Bahreini2014} beschrijft een opstelling waarbij testpersonen achter een computerscherm emoties nadoen om de ML software te trainen. Later focussen ze op de herkenning zelf. In dit onderzoek wordt er gebruik gemaakt van al bestaande software dus enkel het deel na de training is relevant. Ze concluderen dat hun zelfontwikkelde sofware goed werkt voor het meten van de bovenstaande emoties en bruikbaar kan zijn in veel werkomgevingen.

\textcite{Whitehill2014} behandelt een studie over ML software die meet hoe een student geëngageerd is en gefocust is tijdens het werken. Ze zetten ook hun eigen ML software op. Ze concluderen dat ML software gebruikt kan worden voor een 'real-time' automatische engagement detector vergelijkbaar met de accuraatheid van mensen. De data van de software was gecorreleerd met de resultaten van een kennis test achteraf. Dit onderzoek is zeer relevant voor mijn methodologie.

\subsection{Bestaande technologieën}
Bestaande technologieën zijn er genoeg voor dit onderzoek zoals de 'Face API' op Azure die de zes bovenstaande emoties kan lezen of meer toegepaste software die ook het engagement van zijn publiek meet \autocite{Doerrfeld2015}.
%---------- Methodologie ------------------------------------------------------
\section{Methodologie}
\label{sec:methodologie}

%Hier beschrijf je hoe je van plan bent het onderzoek te voeren. Welke onderzoekstechniek ga je toepassen om elk van je onderzoeksvragen te beantwoorden? Gebruik je hiervoor experimenten, vragenlijsten, simulaties? Je beschrijft ook al welke tools je denkt hiervoor te gebruiken of te ontwikkelen.

Om de onderzoeksvragen te beantwoorden zal er gewerkt worden met twee delen. Enerzijds is er een noodzaak aan een vergelijkende studie tussen software om focus te meten op het vlak van emoties. Dit deel tracht een antwoord te vinden op de vraag welke software het meest geschikt is om focus te meten. Als proefopstelling zal een student aan een computer onderwezen worden door videobeelden. Ondertussen wordt hij of zij gefilmd en deze beelden worden onderworpen aan emotieherkenningssoftware. Achteraf wordt de persoon getest over het onderwezen onderwerp en wordt zijn score vergeleken met zijn gemeten focus.

Een tweede deel van het onderzoek omvat het uittesten van een echte proefopstelling in een lokaal waar onderwezen wordt. Hier wordt een onderzoek gedaan op welke manier het mogelijk is om deze opstelling te maken en hoe lectoren hun les kunnen verbeteren op basis van de data de verkozen software genereert. De onderzoekstechnieken hier bestaan uit vragenlijsten of interviews met lectoren over de mogelijkheden van een dergelijke opstelling en experimenten door een ontwikkelde opstelling. Na het gebruik van deze opstelling wordt de lector ondervraagd over de resultaten hiervan.


%---------- Verwachte resultaten ----------------------------------------------
\section{Verwachte resultaten}
\label{sec:verwachte_resultaten}

%Hier beschrijf je welke resultaten je verwacht. Als je metingen en simulaties uitvoert, kan je hier al mock-ups maken van de grafieken samen met de verwachte conclusies. Benoem zeker al je assen en de stukken van de grafiek die je gaat gebruiken. Dit zorgt ervoor dat je concreet weet hoe je je data gaat moeten structureren.


De verwachte resultaten van het eerste deel is een vergelijking tussen de gemeten emoties tijdens het onderwijzen van testpersonen en de antwoorden op de test achteraf. 

Uit de vragenlijsten en interviews verwacht ik antwoorden op de vraag hoe een gemiddelde focusgraad (of trend gedurende een les) kan helpen de les te verbeteren.

De proefopstelling zelf zal antwoorden geven hoe de data van de software gebruikt werd en hoe deze een impact heeft gehad. 


%---------- Verwachte conclusies ----------------------------------------------
\section{Verwachte conclusies}
\label{sec:verwachte_conclusies}

%Hier beschrijf je wat je verwacht uit je onderzoek, met de motivatie waarom. Het is \textbf{niet} erg indien uit je onderzoek andere resultaten en conclusies vloeien dan dat je hier beschrijft: het is dan juist interessant om te onderzoeken waarom jouw hypothesen niet overeenkomen met de resultaten.

Uit het eerste deel van het onderzoek verwacht ik een bepaalde software die het meest geschikt is om de focusgraad van een student te meten. Dit op het gebied van de emoties die de software kan meten en de haalbaarheid om deze te gebruiken in een proefopstelling in een klaslokaal.

Na de proefopstelling kan er besloten worden of het opstellen van een camera en emotieherkennende software lectoren kunnen voorzien van info die daadwerkelijk gebruikt kan worden. Ik verwacht dat een dergelijke opstelling effectief nuttige info kan verschaffen aan lectoren.

%------------------------------------------------------------------------------
% Referentielijst
%------------------------------------------------------------------------------
% de gerefereerde werken moeten in BibTeX-bestand ``voorstel.bib''
% voorkomen. Gebruik JabRef om je bibliografie bij te houden en vergeet niet
% om compatibiliteit met Biber/BibLaTeX aan te zetten (File > Switch to
% BibLaTeX mode)

\phantomsection
\printbibliography[heading=bibintoc]

\end{document}
