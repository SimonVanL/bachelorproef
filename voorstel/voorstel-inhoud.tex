%---------- Inleiding ---------------------------------------------------------

\section{Introductie} % The \section*{} command stops section numbering
\label{sec:introductie}

Hier introduceer je werk. Je hoeft hier nog niet te technisch te gaan.

Je beschrijft zeker:

\begin{itemize}
  \item de probleemstelling en context
  \item de motivatie en relevantie voor het onderzoek
  \item de doelstelling en onderzoeksvraag/-vragen
\end{itemize}

Het aanbrengen van nieuwe leerstof is een continue opdracht voor lectoren. Technologische vooruitgang op het vlak van machineleertechnieken en artificiele intelligentie kunnen inzichten geven over de graad van focus van de studenten. Lectoren kunnen aan de hand van data over de emotionele staat in de klas of een historisch overzicht van voorgaande lessen het aanleren van nieuwe stof optimaliseren. 

Dit onderzoek tracht antwoorden te vinden op de volgende onderzoeksvragen: 
\textbf{Hoe kan gezichtsherkenning van studenten via machineleertechnieken de leskwaliteit verbeteren in het hoger onderwijs?}
\begin{itemize}
    \item Welke software is er geschikt om focus te meten op vlak van emoties?
    \item Hoe kan een lector doormiddel van een trend van de focus in de klas (zowel huidig als historisch) zijn les aanpassen?
\end{itemize}

%---------- Stand van zaken ---------------------------------------------------

\section{State-of-the-art}
\label{sec:state-of-the-art}

Hier beschrijf je de \emph{state-of-the-art} rondom je gekozen onderzoeksdomein. Dit kan bijvoorbeeld een literatuurstudie zijn. Je mag de titel van deze sectie ook aanpassen (literatuurstudie, stand van zaken, enz.). Zijn er al gelijkaardige onderzoeken gevoerd? Wat concluderen ze? Wat is het verschil met jouw onderzoek? Wat is de relevantie met jouw onderzoek?

Verwijs bij elke introductie van een term of bewering over het domein naar de vakliteratuur, bijvoorbeeld! Denk zeker goed na welke werken je refereert en waarom.

% Voor literatuurverwijzingen zijn er twee belangrijke commando's:
% \autocite{KEY} => (Auteur, jaartal) Gebruik dit als de naam van de auteur
%   geen onderdeel is van de zin.
% \textcite{KEY} => Auteur (jaartal)  Gebruik dit als de auteursnaam wel een
%   functie heeft in de zin (bv. ``Uit onderzoek door Doll & Hill (1954) bleek
%   ...'')

Je mag gerust gebruik maken van subsecties in dit onderdeel.

\subsection{Machineleertechnieken en gezichtsherkenning}
De thesis \textcite{Jonsson2018} geeft duiding over het onderzoeksdomein machineleertechnieken en kunstmatige intelligentie. Kunstmatige intelligentie (AI) is concept bestaande van 1950 dat in de praktijk als volgt beschreven kan worden: het automatiseren van taken die menselijk denkwerk nodig hebben. 

Een huidige vorm hiervan zijn machineleertechnieken (ML) waaronder gezichts- en emotieherkenning vallen. De aanpak bij ML bestaat niet uit klassieke geprogrammeerde regels. Een systeem op basis van ML tracht een regelmaat te vinden in een gestructureerde dataset (voor dit onderzoek gefilmde gezichten) en past bekomen regels toe op nieuwe data. 

\subsection{Gerelateerde onderzoeken}

%---------- Methodologie ------------------------------------------------------
\section{Methodologie}
\label{sec:methodologie}

Hier beschrijf je hoe je van plan bent het onderzoek te voeren. Welke onderzoekstechniek ga je toepassen om elk van je onderzoeksvragen te beantwoorden? Gebruik je hiervoor experimenten, vragenlijsten, simulaties? Je beschrijft ook al welke tools je denkt hiervoor te gebruiken of te ontwikkelen.

Om de onderzoeksvragen te beantwoorden zal er gewerkt worden met twee delen. Enerzijds is er een noodzaak aan een vergelijkende studie tussen software om focus te meten op het vlak van emoties. Dit deel tracht een antwoord te vinden op de vraag welke software het meest geschikt is om focus te meten. Als proefopstelling zal een student aan een computer onderwezen worden door videobeelden. Ondertussen wordt hij of zij gefilmd en deze beelden worden onderworpen aan emotieherkenningssoftware. Achteraf vult de testpersoon een vragenlijst in over hoe hij onderwezen is in eenzelfde trend als [referentie naar onderzoek].

Een tweede deel van het onderzoek omvat het uittesten van een echte proefopstelling in een lokaal waar onderwezen wordt. Hier wordt een onderzoek gedaan op welke manier het mogelijk is om deze opstelling te maken en hoe lectoren hun les kunnen verbeteren op basis van de data de verkozen software genereert. De onderzoekstechnieken hier bestaan uit vragenlijsten of interviews met lectoren over de mogelijkheden van een dergelijke opstelling en experimenten door een ontwikkelde opstelling. Na het gebruik van deze opstelling wordt de lector ondervraagd over de resultaten hiervan.


%---------- Verwachte resultaten ----------------------------------------------
\section{Verwachte resultaten}
\label{sec:verwachte_resultaten}

Hier beschrijf je welke resultaten je verwacht. Als je metingen en simulaties uitvoert, kan je hier al mock-ups maken van de grafieken samen met de verwachte conclusies. Benoem zeker al je assen en de stukken van de grafiek die je gaat gebruiken. Dit zorgt ervoor dat je concreet weet hoe je je data gaat moeten structureren.


De verwachte resultaten van het eerste deel is een vergelijking tussen de gemeten emoties tijdens het onderwijzen van testpersonen en de antwoorden op de vragenlijst achteraf. 

Uit de vragenlijsten en interviews verwacht ik antwoorden op de vraag hoe een gemiddelde focusgraad (of trend gedurende een les) kan helpen de les te verbeteren.

De proefopstelling zelf zal antwoorden geven hoe de data van de software gebruikt werd en hoe deze een impact heeft gehad. 


%---------- Verwachte conclusies ----------------------------------------------
\section{Verwachte conclusies}
\label{sec:verwachte_conclusies}

Hier beschrijf je wat je verwacht uit je onderzoek, met de motivatie waarom. Het is \textbf{niet} erg indien uit je onderzoek andere resultaten en conclusies vloeien dan dat je hier beschrijft: het is dan juist interessant om te onderzoeken waarom jouw hypothesen niet overeenkomen met de resultaten.

Uit het eerste deel van het onderzoek verwacht ik een bepaalde software die het meest geschikt is om de focusgraad van een student te meten. Dit op het gebied van de emoties die de software kan meten en de haalbaarheid om deze te gebruiken in een proefopstelling in een klaslokaal.

Na de proefopstelling kan er besloten worden of het opstellen van een camera en emotieherkennende software lectoren kunnen voorzien van info die daadwerkelijk gebruikt kan worden. Ik verwacht dat een dergelijke opstelling effectief nuttige info kan verschaffen aan lectoren.