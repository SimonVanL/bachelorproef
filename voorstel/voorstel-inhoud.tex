%---------- Inleiding ---------------------------------------------------------

\section{Introductie} % The \section*{} command stops section numbering
\label{sec:introductie}

% Hier introduceer je werk. Je hoeft hier nog niet te technisch te gaan.
% Je beschrijft zeker:
% \begin{itemize}
%  \item de probleemstelling en context
%  \item de motivatie en relevantie voor het onderzoek
%  \item de doelstelling en onderzoeksvraag/-vragen
% \end{itemize}

Voor e-commerce platformen (verkoop websites) is een van de grootste uitdagingen het vertrouwen winnen van hun klant, zij het door een goed ontworpen website, overtuigende productpagina's of het hebben van een vertrouwd merk. \autocite{Mary2018} Meer en meer wordt er gekeken naar de snelheid en ontwerp van deze platformen en productafbeeldingen spelen een grote rol hierin. Dit onderzoek tracht een antwoord te geven op de volgende vragen bij klanten van een 'Business-to-Customer' (B2C) perspectief: \textbf{Hoe zorgen productafbeeldingen voor een betere verkoop door klanten op e-commerce platformen?}
\begin{itemize}
	\item Welke factoren beïnvloeden het koopgedrag van klanten online?
	\item Hoe overhaal je een klant om over te gaan van interesse naar aankoop?
	\item Hoe kunnen foto's gecategoriseerd worden en welke foto's genereren meer verkoop?
	\item Welke ratio bestaat er tussen klikken op een product en het effectief kopen?
\end{itemize}

%---------- Stand van zaken ---------------------------------------------------

\section{State-of-the-art}
\label{sec:state-of-the-art}

%Hier beschrijf je de \emph{state-of-the-art} rondom je gekozen onderzoeksdomein. Dit kan bijvoorbeeld een literatuurstudie zijn. Je mag de titel van deze sectie ook aanpassen (literatuurstudie, stand van zaken, enz.). Zijn er al gelijkaardige onderzoeken gevoerd? Wat concluderen ze? Wat is het verschil met jouw onderzoek? Wat is de relevantie met jouw onderzoek?
%
%Verwijs bij elke introductie van een term of bewering over het domein naar de vakliteratuur, bijvoorbeeld! Denk zeker goed na welke werken je refereert en waarom.

% Voor literatuurverwijzingen zijn er twee belangrijke commando's:
% \autocite{KEY} => (Auteur, jaartal) Gebruik dit als de naam van de auteur
%   geen onderdeel is van de zin.
% \textcite{KEY} => Auteur (jaartal)  Gebruik dit als de auteursnaam wel een
%   functie heeft in de zin (bv. ``Uit onderzoek door Doll & Hill (1954) bleek
%   ...'')

%Je mag gerust gebruik maken van subsecties in dit onderdeel.

\subsection{Probleemstellingen bij e-commerce platformen}
Er bestaan eindeloos veel verkoopwebsites, maar niet elke website ervaart hetzelfde succes. \cite{Kaur2017} beschrijft hoe marketing campagnes via het internet kunnen helpen voor naambekendheid en vertrouwen te scheppen bij de klant en waarom mensen online aankopen verkiezen boven het fysieke winkelen. Ook kan er ingespeeld worden op de volgende facetten \autocite{Stringham2010}: 
\begin{itemize}
	\item gebruiksgemak
	\item naambekendheid
	\item inhoud website
	\item grafische vormgeving website
\end{itemize}

Samen met al deze aspecten is het ook belangrijk stil te staan hoe een product (afbeelding, omschrijving en reviews) tot bij de klant geraakt en waarom deze persoon een aankoopbeslissing neemt. \cite{Smaoui2017} geeft ook meer duiding over hoe de juiste afbeeldingen kunnen zorgen voor deze beslissing. Dit onderzoek handelt meer over marketing gerichte campagnes maar de bevindingen kunnen ook meer inzicht geven voor productfoto's zelf.

\subsection{Gerelateerde onderzoeken}
\cite{Mary2018} beschrijft hoe de presentatie van een product een rol speelt bij verkoop, specifiek toegepast op websites. Er wordt ook aangehaald hoe onzekerheid voor een klant een groot risico is en dat daardoor het vertrouwen dat een website uitstraalt impactvol is voor de koop beslissing. 

\cite{Yang2016} handelt eerder over potenti"ele valkuilen en gevaren die kunnen leiden tot operationele ineffectiviteit. Hieruit kan er gekeken worden hoe websites dit probleem aanpakken.

Om meer inzicht te krijgen in koopgedrag van klanten stelt \cite{Li2015} dat er 5 aspecten zijn die de performantie van e-commerce platformen beoordelen.

Deze 3 onderzoeken kaarten een eerder globale visie aan over het koopgedrag en redeneringen van consumenten en gaat voornamelijk nuttig zijn voor een eerste onderzoek. De hoofdonderzoekvraag wordt pas volledig beantwoord na het dieper ingaan op hoe productafbeeldingen verkoop kunnen be"invloeden.
%---------- Methodologie ------------------------------------------------------
\section{Methodologie}
\label{sec:methodologie}

%Hier beschrijf je hoe je van plan bent het onderzoek te voeren. Welke onderzoekstechniek ga je toepassen om elk van je onderzoeksvragen te beantwoorden? Gebruik je hiervoor experimenten, vragenlijsten, simulaties? Je beschrijft ook al welke tools je denkt hiervoor te gebruiken of te ontwikkelen.

Het doelpubliek voor alle onderzoeken zijn potenti"ele klanten van B2C verkoopswebsites.
Het eerste deel van het onderzoek focust zich voornamelijk op het verwerken van data uit vragenlijsten. Een vragenlijst tracht duiding te geven over hoe klanten tot een aankoopbeslissing komen en welke website factoren daar een aandeel in hebben. 
Parallel daarmee worden er een vragenlijst opgesteld waarin er meerdere foto's getoond worden van een bepaald product. Testpersonen worden bevraagd welke foto van het gegeven product hun het meeste zou overtuigen en waarom. Het is de bedoeling deze foto's te gebruiken in een verkoopswebsite van een klant van Synrise Labs. In het geval er geen toegang is tot data van een echte website zal de studie voornamelijk verder handelen over op welke manier de gekozen producten zich onderscheiden van de rest en of er gelijkende aspecten naar boven kunnen komen.
De verkoop van de producten zelf is ook een deel van het onderzoek maar in een laatste fase. 

Een overzicht van het onderzoek staat beschreven in Figuur \ref{gantt}

\begin{figure}
\begin{ganttchart}{1}{12}
	\gantttitle{Verloop Bachelorproef}{12} \\
	\gantttitlelist{42,...,52,1}{1} \\
	\ganttbar{Voorstel}{1}{2} \\
	\ganttlinkedbar{Literatuurstudie}{3}{7} \\
	\ganttlinkedbar{Alg. onderzoek}{8}{11} \\
	\ganttbar{Foto onderzoek}{8}{11} \\
	\ganttmilestone{Finale versie BP}{12} 
	\ganttlink[link mid=0.25]{elem1}{elem3}
	\ganttlink{elem2}{elem4}
	\ganttlink[link mid=0]{elem3}{elem4}
\end{ganttchart}

\caption{Verloop ingedeeld in weken van het jaar: week 42 tot 1}
\label{gantt}
\end{figure}

%---------- Verwachte resultaten ----------------------------------------------
\section{Verwachte resultaten}
\label{sec:verwachte_resultaten}

%Hier beschrijf je welke resultaten je verwacht. Als je metingen en simulaties uitvoert, kan je hier al mock-ups maken van de grafieken samen met de verwachte conclusies. Benoem zeker al je assen en de stukken van de grafiek die je gaat gebruiken. Dit zorgt ervoor dat je concreet weet hoe je je data gaat moeten structureren.

Uit de eerste vragenlijst kan een resultaat komen dat nagaat of theoretisch bepaalde factoren voor online verkoopgedrag ook overeen komen met de realiteit. Dit is eerder een uitbreidend onderzoek naar wat er al gebeurd is en geeft meer zicht over op welke manier klanten overhaald worden om een product te kopen.

Daarnaast wordt er vergeleken of beter scorende productfoto's ook beter verkopen. Aangezien verkoopscijfers heel tijdsafhankelijk zijn worden alle hoeveelheden genormaliseerd naar de verkoopstrend van vorige jaren. 


%---------- Verwachte conclusies ----------------------------------------------
\section{Verwachte conclusies}
\label{sec:verwachte_conclusies}

%Hier beschrijf je wat je verwacht uit je onderzoek, met de motivatie waarom. Het is \textbf{niet} erg indien uit je onderzoek andere resultaten en conclusies vloeien dan dat je hier beschrijft: het is dan juist interessant om te onderzoeken waarom jouw hypothesen niet overeenkomen met de resultaten.

Uit alle resultaten kan er een beeld opgemaakt worden welke factoren klanten het meest be"invloeden om een aankoop te doen. Uit een eerste literatuurstudie lijkt het dat websites met een goede naam, die vlot en intuitief werken en grafisch goed ontworpen zijn. 

Het tweede deel zal uitmaken of er al dan niet een voorkeur is op het gebied van productafbeeldingen. Met voldoende personen die de vragenlijst invullen en als er aan echte verkoopcijfers geraakt wordt verwacht ik een correlatie tussen de verkoop door gebruik van bepaalde foto's en de data.